%-----------------------------------------------------------------------------%
\chapter{\babSatu}
%-----------------------------------------------------------------------------%
Bab ini menjelaskan latar belakang dari rencana penelitian ini, rumusan-rumusan masalah, ruang lingkup pengerjaan, tujuan pengerjaan, serta sistematika penulisan dari laporan yang dibuat.

%-----------------------------------------------------------------------------%
\section{Latar Belakang}
%-----------------------------------------------------------------------------%

% definisi voronoinya perlu di tambah
Diagram Voronoi dari sekumpulan titik adalah diagram yang membagi suatu da-
erah menjadi bagian-bagian yang paling dekat dengan masing-masing titik tersebut \cite{geometric.algebra}. Daerah yang menjadi daerah cakupan dari titik pusat tersebut dihitung berdasarkan jarak Euclidean, yakni panjang garis lurus yang memisahkan dua titik dalam ruang Euclidean \cite{euclidean.distance}.

Diagram Voronoi telah digunakan di berbagai bidang, seperti kesehatan, geometri, teknologi informasi, dan sebagainya. Beberapa contoh aplikasi dari diagram Voronoi diantaranya: persebaran wabah penyakit endemik \cite{cholera}, penentuan rute untuk robot \cite{robot}, kartografi \cite{cartography}, \textit{image clustering} \cite{image.clustering}, daerah cakupan untuk jaringan nirkabel \cite{voronoi.wireless} dan lain-lain.

% bobot tertentu berdasarkan apa, kapasitas atau apa
% bobot dalam paper itu tuh apa? apakah kekuatan dari power supply, atau capacity, bobot menyebabkan jarak tidak euclidean lagi
Xiaojun, dkk dalam \textit{paper}-nya yang berjudul "Distribution Substation Planning Method Based on Weighted Voronoi Diagram" \cite{substation} telah melakukan penelitian tentang metode perencanaan untuk distribusi \textit{substation} dengan menggunakan \textit{weighted Voronoi diagram}. \textit{Weighted Voronoi diagram} adalah diagram Voronoi yang mana titik pusat dari setiap selnya memiliki bobot tertentu. Dengan adanya bobot ini menyebabkan jarak antara titik pusat dengan garis batas suatu daerah menjadi tidak Euclidean lagi, karena garis pemisah antar daerah tidak lagi berupa garis lurus akan tetapi bisa berupa lingkaran atau lengkungan \cite{spatial.tessellations}. Bobot yang dimaksud dalam \textit{paper} tersebut adalah kapasitas dari \textit{substation}.

Dalam perancangan distribusi \textit{hotspot}, masalah yang dihadapi hampir sama dengan penelitian yang dilakukan Xiaojun, dkk \cite{substation}. Contohnya di kampus Universitas Indonesia (UI), Direktorat Sistem dan Teknologi Informasi (DSTI) menghadapi tantangan untuk menentukan titik-titik \textit{hotspot} yang efektif yang dapat menjangkau seluruh wilayah untuk komunitas warga UI. Kampus UI terdiri dari dua wilayah, yaitu di Salemba dan Depok. Sebagian besar fakultas berada di Depok dengan luas lahan mencapai 320 hektar yang 75 persen dari wilayah kampus UI adalah area hijau yang terdiri dari hutan kota dan danau \cite{ui}. 

Di kampus UI terdapat jaringan nirkabel yang saat ini sudah menjangkau seluruh fakultas \cite{hotspot.ui}. Lokasi dari fakultas-fakultas di kampus UI Depok tersebar dalam jarak yang cukup jauh, yang dihubungkan oleh jalan raya. Hal ini menyebabkan wilayah lain di luar fakultas tidak terjangkau oleh jaringan nirkabel.

Metode yang digunakan dalam penelitian yang dilakukan oleh Xiaojun, dkk \cite{substation} menginspirasikan untuk diterapkan dalam masalah \textit{hotspot} di kampus UI. Hal ini dikarenakan (1) jarak antara \textit{hotspot} dengan \textit{device} bisa dianggap sebagai jarak Euclidean (2) kapasitas \textit{coverage} dari tiap \textit{hotspot} berbeda-beda, di mana kapasitas \textit{coverage} (kapasitas di sini adalah kekuatan signal dan bandwidth) ini bisa dianalogikan dengan \textit{weight} pada \textit{weighted Voronoi diagram}. 

Namun terdapat hal yang menjadi kendala dalam penerapan metode yang digunakan Xiaojun, dkk \cite{substation} pada masalah \textit{hotspot}. \textit{Hotspot} memiliki batas jangkauan maksimum yang dapat dia cakup. %source here
Dan bentuk \textit{coverage} dari \textit{hotspot} berupa lingkaran, sehingga memungkinkan adanya daerah yang tidak terjangkau oleh jaringan tersebut. %source again here

Penelitian distribusi \textit{hotspot} akan dilakukan di area kampus UI Depok, Jawa Barat. Penempatan \textit{hotspot} yang tepat dapat memaksimalkan jangkauan dari jaringan nirkabel, %source here
sehingga metode yang digunakan Xiaojun, dkk \cite{substation} menjanjikan untuk diterapkan dalam masalah penentuan lokasi \textit{hotspot}.

Untuk persebaran jaringan \textit{hotspot} ada beberapa hal yang dipertimbangkan, diantaranya adalah apakah setiap wilayah perlu dijangkau oleh jaringan ini? Kemudian bagaimana agar persebaran dari jaringan ini merata untuk setiap wilayah yang perlu dijangkau; tidak ada \textit{blank spot}, dan mengurangi \textit{overlapping area}.

%-----------------------------------------------------------------------------%
\section{Rumusan Masalah}
%-----------------------------------------------------------------------------%

Untuk saat ini \textit{hotspot} sudah terpasang di beberapa titik di kampus UI Depok. Namun untuk meningkatkan efektivitasnya, perlu dilakukan perencanaan dalam penempatan \textit{hotspot}-nya. Secara spesifik rumusan masalah dari penelitian ini adalah sebagai berikut:

\begin{enumerate}
	\item Bagaimana memetakan lokasi \textit{hotspot} saat ini?
	\item Bagaimana menggambarkan \textit{potential user} dari \textit{hotspot}?
	\item Bagaimana menentukan lokasi \textit{hotspot} yang baru?
	\item Bagaimana penempatan \textit{hotspot} berdasarkan kekuatan dan jangkauan setiap \textit{router} untuk mendapatkan area jangkauan yang seluas mungkin?
	\item Bagaimana menentukan titik-titik \textit{hotspot} yang baru dan kapasitas untuk masing-masing \textit{hotspot} yang diperlukan?
	\item Bagaimana penempatan \textit{hotspot} yang baru serta perencanaan kapasitasnya untuk mencapai \textit{coverage} yang semaksimal mungkin?
	\item Bagaimana analisis tersebut jika dikaitkan dengan jangkauan \textit{hotspot} yang terbatas?

\end{enumerate}

%-----------------------------------------------------------------------------%
\section{Tujuan Penelitian}
%-----------------------------------------------------------------------------%

%\textit{which one?}

%Mengimplementasikan metode pencarian distribusi hotspot dengan menggunakan diagram Voronoi

%Mempelajari perancangan pembuatan sistem pencarian distribusi hotspot dengan menggunakan diagram Voronoi
 
%Membuat sistem yang dapat menganalisis distribusi hotspot saat ini serta mencari lokasi hotspot baru yang paling optimal berdasarkan kapasitas dari hotspot untuk menghilangkan blank spot dan meningkatkan kekuatan signal di area tersebut. 
 
%Mencari lokasi hotspot router berdasarkan kekuatan signal dan bandwidth dari router saat ini serta calon router untuk menghilangkan blank spot di area yang dilalui manusia serta meningkatkan signal di area tersebut.

Tujuan dari penelitian ini adalah mempelajari penggunaan \textit{weighted Voronoi diagram} dalam metode perencanaan distribusi \textit{hotspot} dengan memperhatikan jangkauan \textit{hotspot} yang terbatas di lingkungan kampus Universitas Indonesia Depok, Jawa Barat.

%-----------------------------------------------------------------------------%
\section{Batasan Penelitian}
%-----------------------------------------------------------------------------%

Untuk mengurangi kompleksitas dari penelitian ini, maka dianggap daerah yang diteliti memiliki permukaan yang rata. Bukit, lembah, dan gedung diabaikan sehingga tidak ada perhitungan \textit{hotspot} area untuk lantai tertentu di suatu gedung.

%-----------------------------------------------------------------------------%
\section{Sistematika Penulisan Laporan}
%-----------------------------------------------------------------------------%

Sistematika dalam penulisan tesis ini adalah sebagai berikut:
\begin{enumerate}

\item BAB 1 PENDAHULUAN

Bab ini menjelaskan tentang latar belakang dilakukannya penelitian ini, rumusan-rumusan masalahnya, tujuan dari penelitian ini, ruang lingkup pengerjaan, serta sistematika dari laporan yang dibuat.

\item BAB 2 LANDASAN TEORI

Bab ini menjelaskan tentang teori dan konsep yang digunakan dalam penelitian ini, di antaranya teori tentang \textit{weighted Voronoi Diagram}.

\item BAB 3 METODOLOGI PENELITIAN 

Bab ini menjelaskan metode penelitian yang digunakan, yang terdiri dari langkah-langkah yang dibutuhkan dalam penelitian ini.

\item BAB 4 HASIL PENELITIAN

Bab ini berisi tentang hasil penelitian berupa percobaan yang dilakukan dan hasil dari percobaan.

\item BAB 5 PENUTUP

Bab ini berisi kesimpulan dan saran dari penelitian ini.
\end{enumerate}