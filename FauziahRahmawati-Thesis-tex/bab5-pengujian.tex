\chapter{\babLima}

%-----------------------------------------------------------------------------%
\section{Kesimpulan}
%-----------------------------------------------------------------------------%



%-----------------------------------------------------------------------------%
\section{Saran}
%-----------------------------------------------------------------------------%

Penelitian ini dapat dikembangkan untuk pencarian distribusi \textit{hotspot} dalam ruang tiga dimensi. Sehingga persebaran jaringan tidak hanya ada pada bidang datar, tetapi juga bisa dalam bidang ruang. Misalnya untuk gedung-gedung yang terdiri dari beberapa lantai, jaringan yang terhalang oleh tembok, mempertimbangkan kontur tanah, dan lain-lain.

%\textit{Untuk next nya mungkin bisa dicari voronoi diagram di 3D. jadi kalo hotspot di lantai berapa tuh bisa tau juga range dalam bidang ruangnya. brarti kalo di 3D bedanya sih voronoi diagramnya bukan garis tapi bidang. dan untuk weighted voronoi diagram dia bisa berbentuk kurva juga. terus saran buat pengembangannya juga bisa diperhatikan kontur tanah dan gedung. bisa di-detect blank spotnya juga.}

