%-----------------------------------------------------------------------------%
\chapter{\babEnam}
%-----------------------------------------------------------------------------%

Bab ini menjelaskan tentang kesimpulan yang dapat diambil setelah melakukan implementasi untuk \textit{home automation gateway} serta saran  yang dapat penulis sampaikan untuk pengembangan sistem ini selanjutnya.

%-----------------------------------------------------------------------------%
\section{Kesimpulan}

%\textit{Gateway} dapat dibuat dengan menggunakan 

%Zigbee \textit{coordinator} dapat dihubungkan dengan \textit{gateway} dengan menggunakan sensor virtual yang ditanam dalam \textit{server} REST yang 

Dalam tugas akhir ini penulis berhasil mengimplementasikan sebuah \textit{gateway} yang dapat menghubungkan ZigBee \textit{coordinator} dengan \textit{cloud}. Setelah melakukan implementasi \textit{home automation gateway} dengan IoT \textit{cloud service} berbasis ZigBee \textit{network}, penulis dapat mengambil kesimpulan sebagai berikut.

\begin{enumerate}
\item \textit{Gateway} dapat dihubungkan dengan ZigBee \textit{coordinator} melalui REST API. Komunikasi antara \textit{gateway} dengan REST API dilakukan dengan menerjemahkan pesan dari \textit{client} menjadi perintah REST.

\item \textit{Gateway} dapat mengirimkan dan menerima data dari dan ke \textit{cloud} dengan menggunakan protokol MQTT. Pengiriman data dari \textit{gateway} ke \textit{cloud} dilakukan oleh MQTT \textit{client} yang berperan sebagai \textit{publisher}. Sedangkan penerimaan data dari \textit{cloud} ke \textit{gateway} dilakukan oleh MQTT \textit{client} yang berperan sebagai \textit{subscriber}.

\item Data tentang jaringan ZigBee disimpan dalam \textit{server} Mosquitto sehingga informasi tersebut dapat diakses dari mana saja melalui MQTT \textit{client}.

\item Berdasarkan beberapa percobaan yang telah dilakukan, \textit{gateway} telah berjalan dengan baik karena dapat mengendalikan perangkat lampu ZigBee melalui MQTT \textit{client}.

\end{enumerate}

\section{Saran}

Dari implementasi yang telah dilakukan penulis memiliki beberapa saran yang dapat digunakan untuk mengembangkan sistem ini.

\begin{enumerate}

\item Penggunaan protokol MQTT dapat disempurnakan dengan menyimpan \textit{client} yang pernah bergabung dalam jaringan dan topik yang pernah di-\textit{publish} atau di-\textit{subscribe} oleh \textit{client} tersebut. Hal ini ditujukan agar topik dan data dari \textit{client} tersebut masih tersedia dalam \textit{server} jika suatu saat \textit{client} bergabung kembali ke jaringan.

\item Perlu ditambahkan proses pendaftaran atau izin agar tidak semua \textit{client} dapat mengakses informasi atau mengubah keadaan dari jaringan ZigBee.

\item Pembacaan informasi tentang perangkat ZigBee dan pemberian perintah kepada perangkat ZigBee masih terhitung lambat, sehingga perlu dilakukan optimisasi agar proses yang dilakukan dapat berjalan lebih cepat.

\end{enumerate}
