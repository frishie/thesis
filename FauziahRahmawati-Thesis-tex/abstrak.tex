%
% Halaman Abstrak
%
% @author  Andreas Febrian
% @version 1.00
%

\chapter*{Abstrak}

\vspace*{0.2cm}

\noindent \begin{tabular}{l l p{10cm}}
	Nama&: & \penulis \\
	Program Studi&: & \program \\
	Judul&: & \judul \\
\end{tabular} \\ 

\vspace*{0.5cm}

\noindent 
Mempelajari penggunaan \textit{weighted Voronoi diagram} sebagai metode perencanaan distribusi \textit{hotspot} di lingkungan kampus Universitas Indonesia, Depok memerlukan berbagai pertimbangan. Untuk mendapatkan area jangkauan yang seluas mungkin perlu dipertimbangkan lokasi penempatan perangkat \textit{hotspot}, perhitungan masing-masing kapasitas perangkat, dan jangkauan \textit{hotspot} yang terbatas.

\vspace*{0.2cm}

\noindent Kata Kunci: \\ 
\noindent  
\textit{weighted Voronoi diagram, hotspot}\\ 

\newpage